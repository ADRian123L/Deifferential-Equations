\documentclass{article}
\usepackage{amsmath}
\usepackage{amssymb}
\usepackage{enumitem}
\usepackage{theorem}

% Problem statement formatting
\theoremheaderfont{\normalfont\bfseries}
\theorembodyfont{\normalfont}
\newtheorem{problem}{Problem}

\title{Quiz 5}
\author{Adrian Lozada, U71130053}
\date{March 26, 2023}

\begin{document}
    \maketitle
    \newpage

    % Problem 1
    \begin{problem}
    Use the definition of the Laplace transform to find $F(s) = \mathcal{L}\{f(t)\}(s)$ if:
    \[
        f(t) = 
        \begin{cases}
            t & \text{if } 0 < t < 1 \\
            1 & \text{if } t > 1 \\
        \end{cases}
    \]
    \end{problem}
    
    \textbf{Solution:} \\
    \begin{align*}
        \mathcal{L}\{f\}(s) &= \int_0^1 te^{-st} \, dt + \int_1^\infty e^{-st} \, dt \\
        \\ % Add some vertical space
        \textbf{LHS integral:} \\
        &= \int_0^1 te^{-st} \, dt = -\frac{te^{-st}}{s} - \frac{e^{-st}}{s^2} \Big|_0^1 \\
        &= \left(-\frac{1}{s} - \frac{1}{s^2}\right)e^{-s} - \left(0 - \frac{1}{s^2}\right) \\
        &= \frac{1}{s^2} - \frac{1}{s}e^{-s} - \frac{1}{s^2}e^{-s}, \text{ for } s > 0 \\
        \\ % Add some vertical space
        \textbf{RHS integral:} \\
        &= \int_1^\infty e^{-st} \, dt = -\frac{e^{-st}}{s} \Big|_1^\infty \\
        &= 0 - \left(-\frac{1}{s}e^{-s}\right) \\
        &= \frac{1}{s}e^{-s}, \text{ for } s > 0
    \end{align*}
    \\ % Add some vertical space
    \textbf{Answer:}
    \begin{align*}
       \mathcal{L}\{f\}(s) &= \frac{1}{s^2} - \frac{1}{s}e^{-s} - \frac{1}{s^2}e^{-s} + \frac{1}{s}e^{-s} \\
       &= \frac{1}{s^2} - \frac{1}{s^2}e^{-s} \text{ for } s > 0
    \end{align*} 

    \newpage
    % Problem 2
    \begin{problem}
        Use the definition of the Laplace transform to find $F(s) = \mathcal{L}\{f(t)\}(s)$ if:
        \[
            f(t) = 
            \begin{cases}
                \cos{6t} & \text{if } 0 < t < \frac{\pi}{2} \\
                0 & \text{if } t > \frac{\pi}{2} \\
            \end{cases}
        \]
    \end{problem}
    
    \textbf{Solution:} \\
    \begin{align*}
        \mathcal{L}\{f\}(s) &= \int_{0}^{\frac{\pi}{2}} e^{-st}\cos{6t} \, dt + \int_{\frac{\pi}{2}}^{\infty} 0 \times e^{-st} \, dt \\
    \end{align*}
    The second integral is zero since it is a product of 0 and an exponential function. We only need to solve the first integral:
    \begin{align*}
        \mathcal{L}\{f\}(s) &= \int_{0}^{\frac{\pi}{2}} e^{-st}\cos{6t} \, dt \\
    \end{align*}
    Now, we integrate by parts again. Let $u = \cos{6t}$ and $dv = e^{-st}\, dt$. \\
    Then, $du = -6\sin{6t}\, dt$ and $v = -\frac{1}{s}e^{-st}$.

    \begin{align*}
        \mathcal{L}\{f\}(s) &= \left[ -\frac{1}{s}e^{-st}\cos{6t}\right]_{0}^{\frac{\pi}{2}} - \frac{6}{s}\int_{0}^{\frac{\pi}{2}} \frac{1}{s}e^{-st}\sin{6t}\, dt \\   
    \end{align*}
    Now, we integrate by parts again. Let $u = \sin{6t}$ and $dv = e^{-st}\, dt$. \\
    Then, $du = 6\cos{6t}\, dt$ and $v = -\frac{1}{s}e^{-st}$. \\
    \begin{align*}
        \mathcal{L}\{f\}(s) &= \left[ -\frac{1}{s}e^{-st}\cos{6t}\right]_{0}^{\frac{\pi}{2}} - \left[ -\frac{6}{s^{2}}e^{-st}\sin{6t}\right]_{0}^{\frac{\pi}{2}} - \frac{36}{s^{2}}\mathcal{L}\{f\}(s)
    \end{align*}    
    Now, we add $\frac{36}{s^{2}}\mathcal{L}\{f\}(s)$ to both sides of the equation. \\
    Then we factor out $\mathcal{L}\{f\}(s)$. \\
    \begin{align*}
        \mathcal{L}\{f\}(s)\left(\frac{s^{2} + 36}{s^{2}}\right) &= \left[ -\frac{1}{s}e^{-st}\cos{6t}\right]_{0}^{\frac{\pi}{2}} - \left[ -\frac{6}{s^{2}}e^{-st}\sin{6t}\right]_{0}^{\frac{\pi}{2}} \\
        &= \left( \left(-\frac{1}{s}e^{-s\frac{\pi}{2}} + \frac{1}{s}\right) + \left(0\right)\right) 
    \end{align*}
    Now, we multiply both sides by $\frac{s^{2}}{s^{2} + 36}$ and simplify. \\
    \begin{align*}
        \mathcal{L}\{f\}(s) &= \frac{s^{2}}{s^{2} + 36}\left(-\frac{1}{s}e^{-s\frac{\pi}{2}} + \frac{1}{s}\right) \\
        &= \frac{s}{s^{2} + 36}(-e^{-s\frac{\pi}{2}} + 1) \\
        &= \frac{s(1 - e^{-s\frac{\pi}{2}})}{s^{2} + 36}, \text{ for } s > 0 \\
    \end{align*}

    \newpage
    % Problem 3
    \begin{problem}
        Use the Laplace Transform Table (LTT) to find F(s) = $\mathcal{L}\{f\}(s)$ if:
        \[
            f(t) = (\sin{t} + \cos{t})^{2}
        \]
    \end{problem}
    \textbf{Solution:}
    \begin{align*}
        f(t) &= \sin^{2}{t} +2\sin{t}\cos{t} + \cos^{2}{t} \\
        &= \frac{1}{2} - \frac{1}{2}\cos{2t} + \sin{2t} + \frac{1}{2} + \frac{1}{2}\cos{2t} \\
        &= 1 + \sin{2t} 
    \end{align*}
    Now, we use the LTT to find the Laplace transform of $1 + \sin{2t}$. \\
    \begin{align*}
        \mathcal{L}\{f\}(s) &= \mathcal{L}\{1\} + \mathcal{L}\{\sin{2t}\} \\
        &= \frac{1}{s} + \frac{1}{s^{2} + 4}, \text{ for } s > 0 
    \end{align*}

    \newpage
    % Problem 4
    \begin{problem}
        Use the Laplace Transform Table (LTT) to find F(s) = $\mathcal{L}\{f\}(s)$ if:
        \[
            f(t) = t^{4} + t^{2}e^{-t} + \sin^{2}{t} - e^{3t}\sin{\sqrt{3}t}
        \]
    \end{problem}
    \textbf{Solution:}
    \begin{align*}
        &= \mathcal{L}\{t^{4}\}  + \mathcal{L}\{t^{2}e^{-t}\} + \mathcal{L}\{\sin^{2}{t}\} + \mathcal{L}\{e^{3t}\sin{\sqrt{3}t}\} \\
        &= \mathcal{L}\{t^{4}\}  + \mathcal{L}\{t^{2}e^{-t}\} + \frac{1}{2}\mathcal{L}\{1\} - \frac{1}{2}\mathcal{L}\{\cos{2t}\} - \mathcal{L}\{e^{3t}\sin{\sqrt{3}t}\} \\
        &= \frac{4!}{s^{5}} + \frac{2!}{(s + 1)^{3}} + \frac{1}{2s} - \frac{1}{2}\frac{s}{s^{2} + 4} -\frac{\sqrt{3}}{(s - 3)^{2} + 3} \\
        &= \frac{24}{s^{5}} + \frac{2}{(s + 1)^{3}} + \frac{1}{2s} - \frac{1}{2}\frac{s}{s^{2} + 4} -\frac{\sqrt{3}}{(s - 3)^{2} + 3} \\
    \end{align*}

\end{document}
