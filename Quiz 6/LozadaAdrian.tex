\documentclass{article}
\usepackage{amsmath}
\usepackage{amssymb}
\usepackage{theorem}
\usepackage{enumitem}

\title{Quiz 6}
\author{Adrian Lozada, U71130053}
\date{April 2, 2023}
\theoremheaderfont{\normalfont\bfseries}
\theorembodyfont{\normalfont}
\newtheorem{problem}{Problem}

\begin{document}
    \maketitle
    \newpage

    % Problem 1
    \begin{problem}
        Find the inverse Laplace transform of the given function.
        \[            \frac{2s + 16}{s^{2} + 4s + 13}        \]
    \end{problem}
    \textbf{Solution:} \\ \\ 
    \textit{Step 1: Simplify the given rational function.} \\
     \begin{align*}
        \frac{2s + 16}{s^{2} + 4s + 13} &= \frac{2(s + 8)}{(s + 2)^{2} + 9} \\
        &= \frac{2(s + 2 + 6)}{(s + 2)^{2} + (3)^{2}} = \frac{2(s + 2)}{(s + 2)^{2} + (3)^{2}} + \frac{12}{(s + 2)^{2} + (3)^{2}} \\
        &= 2\left( \frac{s + 2}{(s + 2)^{2} + (3)^{2}}\right) + 4\left( \frac{3}{(s + 2)^{2} + (3)^{2}}\right)
     \end{align*}
    \textit{Step 2: Find the inverse Laplace transform.} \\
    \begin{align*}
       &= 2\mathcal{L}^{-1} \left\{ \frac{s + 2}{(s + 2)^{2} + (3)^{2}}\right\} + 4\mathcal{L}^{-1} \left\{ \frac{3}{(s + 2)^{2} + (3)^{2}}\right\} \\ \\
       &= 2(e^{-2t}\cos{3t}) + 4(e^{-2t}\sin{3t}) \\ \\
       &= 2e^{-2t}\cos{3t} + 4e^{-2t}\sin{3t}
    \end{align*}
    \newpage


    % Problem 2
    \begin{problem}
        Determine the partial fraction expansion for the given rational function. 
        \[
            \frac{4s^{2} - 21s + 16}{s(s - 2)^{2}}
        \]
    \end{problem}
    \textbf{Solution:} \\ \\
    \textit{We first simplify the given rational function.} \\
    \begin{align*}
        &= \frac{4s^{2} - 21s + 16}{s(s - 2)^{2}} = \frac{A}{s} + \frac{B}{s - 2} + \frac{C}{(s - 2)^{2}} \\ \\
        &= 4s^{2} - 21s + 16 = A(s - 2)^{2} + Bs(s - 2) + Cs \\ \\
    \end{align*}
    \textit{Now we find the coefficients.} \\
    \begin{align*}
        s = 2: \\
        4(2)^{2} - 21(2) + 16 &= A(0)^{2} + B(0) + 2C \\
        16 - 42 + 16 &= 2C \\
        2C &= -10 \\
        C &= -5 \\ \\
        s = 0: \\
        4(0)^{2} - 21(0) + 16 &= A(2)^{2} \\
        16 &= A(4) \\
        A &= 4 \\ \\
    \end{align*}
    \textit{Solving by equating coefficients.} \\
    \begin{align*}
        &= A(s^{2} - 4s + 4) + B(s^{2} - 2s) + Cs \\
        &= As^{2} - 4As + 4A + Bs^{2} - 2Bs + Cs \\
        &= s^{2}(A + B) +  s(-4A - 2B + C) + 4A \\
    \end{align*}
    \newpage
    \textit{Hence:} \\
    \begin{align*}
        A + B &= 4 \\
        -4A - 2B + C &= -21 \\
        4A &= 16 \\
    \text{Subsitute $A = 4$} \\
        4 + B &= 4 \\
        B &= 0 \\
    \end{align*} 
    \textit{Therefore,} \\
    \begin{align*}
        \frac{4s^{2} -21s + 16}{s(s - 2)^{2}} &= \frac{4}{s} + \frac{0}{s - 2} + \frac{-5}{(s - 2)^{2}} \\
        &= \frac{4}{s} - \frac{5}{(s - 2)^{2}}
    \end{align*}

    \newpage
% Problem 3
    \begin{problem}
        Determine the partial fraction expansion for the given rational function. 
        \[
            \frac{3s + 5}{s(s^{2} + s - 6)}
        \]
    \end{problem}
    \textbf{Solution:} \\ \\
    \textit{We first simplify the given rational function.} \\
    \begin{align*}
     \frac{3s + 5}{s(s^{2} + s - 6)} &= \frac{3s + 5}{s(s + 3)(s - 2)} \\
    \end{align*}
    \textit{Now we find the partial fraction expansion.} \\
    \begin{align*}
        \frac{3s + 5}{s(s + 3)(s - 2)} &= \frac{A}{s} + \frac{B}{s + 3} + \frac{C}{s - 2} \\
        &= A(s + 3)(s - 2) + Bs(s - 2) + Cs(s + 3) \\
    \end{align*}
    \textit{Solving for coefficients.} \\
    \begin{align*}
        s = 0: \\
        3(0) + 5 &= A(3)(-2) + B(0)(0) + C(0)(0) \\
        5 &= -6A \\
        A &= -\frac{5}{6} \\ \\
        s = -3: \\
        3(-3) + 5 &= A(0) + B(-3)(-3 -2) + C(0) \\
        -4 &= B(15) \\
        B &= -\frac{4}{15} \\ \\
        s = 2: \\
        3(2) + 5 &= A(0) + B(0) + C(2)(2 + 3) \\
        11 &= C(10) \\
        C &= \frac{11}{10} \\ \\
    \end{align*}
    \textit{Therefore,} \\
    \begin{align*}
        \frac{3s + 5}{s(s^{2} + s - 6)} &= \frac{-\frac{5}{6}}{s} + \frac{-\frac{4}{15}}{s + 3} + \frac{\frac{11}{10}}{s - 2} \\ \\
        &= -\frac{5}{6s} - \frac{4}{15(s + 3)} + \frac{11}{10(s - 2)}
     \end{align*} 
    \newpage

    % Problem 4
    \begin{problem}
        Determine $\mathcal{L}^{-1}\{F\}$ 
        \[
            F(s) = \frac{5s^{2} + 34s + 53}{(s + 3)^{2}(s + 1)}
        \]
    \end{problem}
    \textbf{Solution:} \\ \\
    \textit{We first simplify the given rational function.} \\
    \begin{align*}
        &= \frac{5s^{2} + 34s + 53}{(s + 3)^{2}(s + 1)} = \frac{A}{s + 3} + \frac{B}{(s + 3)^{2}} + \frac{C}{s + 1} \\ \\
        &= 5s^{2} + 34s + 53 = A(s + 3)(s + 1) + B(s + 1) + C(s + 3)^{2} \\ \\
    \end{align*}
    \textit{Now we find the coefficients.} \\
    \begin{align*}
        s = -3: \\
        5(-3)^{2} + 34(-3) + 53 &= -2B \\
        -4 &= -2B \\
        B &= 2 \\ \\ 
        s = -1: \\
        5(-1)^{2} + 34(-1) + 53 &=  C(2)^{2} \\
        24 &= 4C \\
        C &= 6 \\ \\
    \end{align*}
    \textit{Solving by equating coefficients.} \\
    \begin{align*}
        &=A(s + 3)(s + 1) + B(s + 1) + C(s + 3)^{2} \\ \\
        &= A(s^{2} + 4s + 3) + B(s + 1) + C(s^{2} + 6s + 9) \\ \\
        &= s^{2}(A + C) + s(4A + 6C + B) + 3A + B + 9C \\ \\
        &s^{2}: \\
        A + C &= 5 \\
        A &= 5 - C \\ 
        A &= 5 - 6 = -1 \\
        A &= -1 \\ \\
    \end{align*} 
    \textit{Therefore,} \\
    \begin{align*}
        \frac{5s^{2} + 34s + 53}{(s + 3)^{2}(s + 1)} &= -\frac{1}{s + 3} + \frac{2}{(s + 3)^{2}} + \frac{6}{(s + 1)} \\ \\
    \end{align*}
    \textit{Now we find the inverse Laplace transform.} \\
    \begin{align*}
        \mathcal{L}^{-1}\{F\} &= -\mathcal{L}^{-1} \left\{ \frac{1}{s + 3}\right\} + 2\mathcal{L}^{-1} \left\{ \frac{1}{(s + 3)^{2}}\right\} + 6\mathcal{L}^{-1} \left\{ \frac{1}{(s + 1)}\right\} \\ \\
        &= -e^{-3t} + 2te^{-3t} + 6e^{-t} \\ \\
    \end{align*}
\end{document}